\documentclass[12pt,a4paper]{article}

% Load packages BEFORE xepersian
\usepackage{geometry}
\usepackage{xcolor}
\usepackage{amsmath}
\usepackage{amsfonts}
\usepackage{graphicx}
\usepackage{booktabs}
\usepackage{longtable}
\usepackage{array}
\usepackage{fancyhdr}
\usepackage{titlesec}
\usepackage{tcolorbox}
\usepackage{hyperref}

% Listings package with custom setup
\usepackage{listings}

% XePersian setup for RTL/LTR support - MUST BE LOADED LAST
\usepackage{xepersian}
\settextfont[Scale=1.0]{IRANSans}
\setlatintextfont[Scale=1.0]{Times New Roman}
\DefaultMathsDigits

% Page setup
\geometry{
	top=2.5cm,
	bottom=2.5cm,
	left=2.5cm,
	right=2.5cm
}

% Colors
\definecolor{codegreen}{rgb}{0,0.6,0}
\definecolor{codegray}{rgb}{0.5,0.5,0.5}
\definecolor{codepurple}{rgb}{0.58,0,0.82}
\definecolor{backcolour}{rgb}{0.95,0.95,0.92}
\definecolor{sectioncolor}{rgb}{0.2,0.4,0.6}

% Special environment for LTR code blocks
\newenvironment{ltrcode}{\lr\bgroup}{\egroup}

% Code styling with forced LTR
\lstdefinestyle{pythonstyle}{
	backgroundcolor=\color{backcolour},   
	commentstyle=\color{codegreen},
	keywordstyle=\color{magenta},
	numberstyle=\tiny\color{codegray},
	stringstyle=\color{codepurple},
	basicstyle=\footnotesize\ttfamily,
	breakatwhitespace=false,         
	breaklines=true,                 
	captionpos=b,                    
	keepspaces=true,                 
	numbers=left,                    
	numbersep=5pt,                  
	showspaces=false,                
	showstringspaces=false,
	showtabs=false,                  
	tabsize=2,
	frame=single,
	rulecolor=\color{black},
	upquote=true,
	columns=fullflexible
}

\lstset{style=pythonstyle}

% Custom command for Python code with LTR direction
\newcommand{\pythoncode}[2][]{%
	\begin{ltrcode}%
		\lstinputlisting[language=Python,#1]{#2}%
	\end{ltrcode}%
}

% Header and footer
\pagestyle{fancy}
\fancyhf{}
\fancyhead[R]{\lr{Genetic Algorithm for Equation Solving}}
\fancyhead[L]{حل‌کننده ژنتیک سیستم معادلات}
\fancyfoot[C]{\thepage}

% Section styling
\titleformat{\section}
{\color{sectioncolor}\Large\bfseries}
{\thesection}{1em}{}

\titleformat{\subsection}
{\color{sectioncolor}\large\bfseries}
{\thesubsection}{1em}{}

% Hyperlink setup
\hypersetup{
	colorlinks=true,
	linkcolor=blue,
	filecolor=magenta,      
	urlcolor=cyan,
	pdfpagemode=FullScreen,
	unicode=true,
	pdfencoding=auto
}

% Document title
\title{\Huge\bfseries راهنمای کامل حل‌کننده ژنتیک سیستم معادلات\\
	\vspace{0.5cm}
	\Large\lr{Complete Guide to Genetic Algorithm Equation Solver}}

\author{مستندات فنی الگوریتم ژنتیک}
\date{\today}

\begin{document}
	
	\maketitle
	
	\tableofcontents
	\newpage
	
	\section{مقدمه}
	
	این مستند راهنمای کاملی برای درک و استفاده از حل‌کننده ژنتیک سیستم معادلات ارائه می‌دهد. این سیستم قادر به حل انواع مختلف معادلات خطی و غیرخطی با استفاده از الگوریتم ژنتیک پیشرفته است.
	
	\subsection{ویژگی‌های کلیدی}
	\begin{itemize}
		\item حل سیستم‌های معادلات خطی و غیرخطی
		\item استفاده از استراتژی‌های متنوع جهش و ترکیب
		\item پشتیبانی از تابع شایستگی تطبیقی
		\item قابلیت راه‌اندازی مجدد خودکار
		\item تبدیل اعداد اعشاری به کسرهای دقیق
	\end{itemize}
	
	\section{تابع شایستگی}
	
	\subsection{چرا از این تابع شایستگی استفاده شد؟}
	
	\begin{ltrcode}
		\begin{lstlisting}[language=Python, caption=تابع شایستگی اصلی]
			def fitness_function(self, chromosome, equations):
			total_error = 0
			for eq in equations:
			try:
			result = eq(*chromosome)
			total_error += result**2  # خطای مربعی
			except:
			total_error += 1e6
			
			if total_error < 1e-15:
			return float('inf')
			
			return 1.0 / total_error
		\end{lstlisting}
	\end{ltrcode}
	
	\subsection{دلایل انتخاب این روش:}
	
	\subsubsection{استفاده از خطای مربعی (\lr{\texttt{result**2}})}
	\begin{itemize}
		\item \textbf{دلیل}: خطاهای بزرگ بیشتر تنبیه می‌شوند
		\item \textbf{مزیت}: همگرایی سریع‌تر به جواب دقیق
		\item \textbf{مثال}: اگر دو کروموزوم خطاهای 0.1 و 0.01 داشته باشند:
		\begin{itemize}
			\item روش خطی: 0.1 vs 0.01 (تفاوت 10 برابری)
			\item روش مربعی: 0.01 vs 0.0001 (تفاوت 100 برابری)
		\end{itemize}
	\end{itemize}
	
	\subsubsection{معکوس خطا (\lr{\texttt{1.0 / total\_error}})}
	\begin{itemize}
		\item \textbf{دلیل}: تبدیل مسئله minimization به maximization
		\item \textbf{مزیت}: الگوریتم ژنتیک معمولاً برای بیشینه‌سازی طراحی شده
		\item \textbf{نتیجه}: خطای کمتر = شایستگی بیشتر
	\end{itemize}
	
	\subsubsection{مدیریت استثناء}
	\begin{itemize}
		\item \textbf{جریمه سنگین} (\lr{\texttt{1e6}}) برای کروموزوم‌های نامعتبر
		\item \textbf{جلوگیری} از تقسیم بر صفر، سرریز عددی، و خطاهای محاسباتی
	\end{itemize}
	
	\subsubsection{تشخیص جواب دقیق}
	\begin{itemize}
		\item اگر خطا کمتر از \lr{\texttt{1e-15}} باشد، شایستگی \lr{\texttt{inf}} می‌شود
		\item \textbf{نتیجه}: توقف زودهنگام در صورت یافتن جواب دقیق
	\end{itemize}
	
	\section{روش‌های جایگزین تابع شایستگی}
	
	\subsection{روش خطای خطی (\lr{Linear Error})}
	
	\begin{ltrcode}
		\begin{lstlisting}[language=Python, caption=تابع شایستگی خطی]
			def linear_fitness(self, chromosome, equations):
			total_error = 0
			for eq in equations:
			total_error += abs(eq(*chromosome))
			return 1.0 / (1.0 + total_error)
		\end{lstlisting}
	\end{ltrcode}
	
	\textbf{مزایا:}
	\begin{itemize}
		\item ساده‌تر در پیاده‌سازی
		\item حساسیت کمتر به نویز
		\item پایداری عددی بهتر
	\end{itemize}
	
	\textbf{معایب:}
	\begin{itemize}
		\item همگرایی کندتر
		\item تمایز کمتر بین جواب‌های خوب و عالی
		\item ممکن است در جواب‌های تقریبی گیر کند
	\end{itemize}
	
	\subsection{روش خطای لگاریتمی (\lr{Logarithmic Error})}
	
	\begin{ltrcode}
		\begin{lstlisting}[language=Python, caption=تابع شایستگی لگاریتمی]
			def log_fitness(self, chromosome, equations):
			total_error = 0
			for eq in equations:
			error = abs(eq(*chromosome))
			if error > 0:
			total_error += math.log(1 + error)
			return 1.0 / (1.0 + total_error)
		\end{lstlisting}
	\end{ltrcode}
	
	\textbf{مزایا:}
	\begin{itemize}
		\item کاهش اثر خطاهای بزرگ
		\item تعادل بهتر بین دقت و پایداری
		\item مناسب برای مسائل با مقیاس‌های مختلف
	\end{itemize}
	
	\textbf{معایب:}
	\begin{itemize}
		\item پیچیدگی محاسباتی بیشتر
		\item ممکن است دقت نهایی کمتری داشته باشد
		\item نیاز به تنظیم پارامترهای اضافی
	\end{itemize}
	
	\subsection{روش وزن‌دار (\lr{Weighted Error})}
	
	\begin{ltrcode}
		\begin{lstlisting}[language=Python, caption=تابع شایستگی وزن‌دار]
			def weighted_fitness(self, chromosome, equations, weights=None):
			if weights is None:
			weights = [1.0] * len(equations)
			
			total_error = 0
			for eq, weight in zip(equations, weights):
			error = eq(*chromosome)
			total_error += weight * (error ** 2)
			
			return 1.0 / (1.0 + total_error)
		\end{lstlisting}
	\end{ltrcode}
	
	\textbf{مزایا:}
	\begin{itemize}
		\item امکان اولویت‌بندی معادلات مختلف
		\item انعطاف‌پذیری بالا
		\item کنترل بهتر بر فرآیند همگرایی
	\end{itemize}
	
	\textbf{معایب:}
	\begin{itemize}
		\item نیاز به تعیین وزن‌های مناسب
		\item پیچیدگی بیشتر در تنظیم
		\item ممکن است به برخی معادلات بی‌توجهی کند
	\end{itemize}
	
	\subsection{روش نرمالایز شده (\lr{Normalized Error})}
	
	\begin{ltrcode}
		\begin{lstlisting}[language=Python, caption=تابع شایستگی نرمالایز شده]
			def normalized_fitness(self, chromosome, equations):
			errors = []
			for eq in equations:
			errors.append(abs(eq(*chromosome)))
			
			# نرمالایز بر اساس میانگین و انحراف معیار
			mean_error = sum(errors) / len(errors)
			if mean_error == 0:
			return float('inf')
			
			normalized_error = sum(e / mean_error for e in errors)
			return 1.0 / (1.0 + normalized_error)
		\end{lstlisting}
	\end{ltrcode}
	
	\textbf{مزایا:}
	\begin{itemize}
		\item تعادل بهتر بین معادلات مختلف
		\item کاهش اثر معادلات با مقیاس بزرگ
		\item عملکرد یکنواخت‌تر
	\end{itemize}
	
	\textbf{معایب:}
	\begin{itemize}
		\item محاسبات اضافی برای نرمالایز
		\item ممکن است اطلاعاتی از دست برود
		\item پیچیدگی در پیاده‌سازی
	\end{itemize}
	
	\subsection{روش تطبیقی (\lr{Adaptive Fitness})}
	
	\begin{ltrcode}
		\begin{lstlisting}[language=Python, caption=تابع شایستگی تطبیقی]
			def adaptive_fitness(self, chromosome, equations, generation):
			total_error = 0
			
			# ضریب تطبیقی بر اساس نسل
			adaptive_power = 2.0 + (generation / 1000.0)
			
			for eq in equations:
			error = abs(eq(*chromosome))
			total_error += error ** adaptive_power
			
			return 1.0 / (1.0 + total_error)
		\end{lstlisting}
	\end{ltrcode}
	
	\textbf{مزایا:}
	\begin{itemize}
		\item تطبیق خودکار با پیشرفت الگوریتم
		\item شروع با جستجوی گسترده، ادامه با جستجوی دقیق
		\item بهینه‌سازی مراحل مختلف حل
	\end{itemize}
	
	\textbf{معایب:}
	\begin{itemize}
		\item پیچیدگی زیاد
		\item نیاز به تنظیم پارامترهای بیشتر
		\item ممکن است پیش‌بینی‌ناپذیر باشد
	\end{itemize}
	
	\subsection{روش ترکیبی چندگانه (\lr{Multi-Objective})}
	
	\begin{ltrcode}
		\begin{lstlisting}[language=Python, caption=تابع شایستگی چندهدفه]
			def multi_objective_fitness(self, chromosome, equations):
			# هدف 1: کمینه‌سازی خطا
			total_error = sum(eq(*chromosome)**2 for eq in equations)
			
			# هدف 2: سادگی جواب (ترجیح اعداد کوچک)
			complexity = sum(abs(x) for x in chromosome)
			
			# هدف 3: تعداد کسرهای ساده
			fraction_score = sum(1 for x in chromosome 
			if abs(x - round(x*4)/4) < 1e-3)
			
			# ترکیب وزن‌دار
			fitness = (1.0 / (1 + total_error)) * 0.7 + \
			(1.0 / (1 + complexity)) * 0.2 + \
			fraction_score * 0.1
			
			return fitness
		\end{lstlisting}
	\end{ltrcode}
	
	\textbf{مزایا:}
	\begin{itemize}
		\item در نظر گیری چندین معیار همزمان
		\item تولید جواب‌های عملی‌تر
		\item انعطاف‌پذیری بسیار بالا
	\end{itemize}
	
	\textbf{معایب:}
	\begin{itemize}
		\item پیچیدگی زیاد در طراحی و تنظیم
		\item ممکن است همگرایی آهسته‌تر داشته باشد
		\item نیاز به دانش تخصصی برای تنظیم وزن‌ها
	\end{itemize}
	
	\section{جمع‌بندی تابع شایستگی}
	
	\subsection{چرا روش فعلی (خطای مربعی معکوس) بهترین انتخاب است؟}
	
	\begin{enumerate}
		\item \textbf{سادگی و کارایی}: پیاده‌سازی ساده با عملکرد عالی
		\item \textbf{همگرایی سریع}: خطاهای بزرگ شدیداً تنبیه می‌شوند
		\item \textbf{دقت بالا}: قادر به یافتن جواب‌های بسیار دقیق
		\item \textbf{پایداری}: مدیریت مناسب حالات استثنایی
	\end{enumerate}
	
	\subsection{کی از روش‌های دیگر استفاده کنیم؟}
	
	\begin{itemize}
		\item \textbf{خطای خطی}: برای مسائل حساس به نویز
		\item \textbf{وزن‌دار}: وقتی برخی معادلات اهمیت بیشتری دارند
		\item \textbf{نرمالایز}: برای معادلات با مقیاس‌های مختلف
		\item \textbf{تطبیقی}: برای مسائل خیلی پیچیده
		\item \textbf{چندهدفه}: وقتی کیفیت جواب اهمیت دارد
	\end{itemize}
	
	این انتخاب بستگی به نوع مسئله، دقت مورد نیاز، و منابع محاسباتی در دسترس دارد.
	
	\section{عوامل مؤثر در الگوریتم ژنتیک}
	
	\subsection{فهرست عوامل به ترتیب اولویت}
	
	\begin{enumerate}
		\item روش انتخاب (\lr{Selection Strategy})
		\item عملگر ترکیب (\lr{Crossover Operator})
		\item استراتژی جهش (\lr{Mutation Strategy})
		\item مدیریت جمعیت (\lr{Population Management})
		\item پارامترهای کنترلی (\lr{Control Parameters})
		\item نحوه ایجاد جمعیت اولیه (\lr{Initialization Strategy})
		\item معیارهای توقف (\lr{Termination Criteria})
	\end{enumerate}
	
	\subsection{روش انتخاب (\lr{Selection Strategy})}
	
	\subsubsection{تأثیر: \textbf{بالا} (اولویت 1)}
	\textbf{دلیل}: تعیین‌کننده نحوه انتشار ژن‌های خوب در جمعیت
	
	\subsubsection{روش استفاده شده در کد:}
	
	\begin{ltrcode}
		\begin{lstlisting}[language=Python, caption=Tournament Selection]
			def selection(self, ranked_population):
			tournament = random.sample(ranked_population, self.tournament_size)
			if random.random() < 0.9:
			return max(tournament, key=lambda x: x[1])[0]
			else:
			sorted_tournament = sorted(tournament, key=lambda x: x[1], reverse=True)
			return sorted_tournament[1][0] if len(sorted_tournament) > 1 else sorted_tournament[0][0]
		\end{lstlisting}
	\end{ltrcode}
	
	\textbf{نوع}: Tournament Selection با اندازه 5
	
	\subsubsection{روش‌های جایگزین:}
	
	\paragraph{1.1. Roulette Wheel Selection}
	
	\begin{ltrcode}
		\begin{lstlisting}[language=Python, caption=Roulette Wheel Selection]
			def roulette_wheel_selection(self, ranked_population):
			total_fitness = sum(fitness for _, fitness in ranked_population)
			pick = random.uniform(0, total_fitness)
			current = 0
			for chromosome, fitness in ranked_population:
			current += fitness
			if current > pick:
			return chromosome
		\end{lstlisting}
	\end{ltrcode}
	
	\textbf{مزایا:}
	\begin{itemize}
		\item احتمال انتخاب متناسب با شایستگی
		\item حفظ تنوع بهتر
		\item پیاده‌سازی ساده
	\end{itemize}
	
	\textbf{معایب:}
	\begin{itemize}
		\item حساس به scaling مقادیر شایستگی
		\item ممکن است افراد ضعیف زیاد انتخاب شوند
		\item مشکل با شایستگی‌های منفی
	\end{itemize}
	
	\paragraph{1.2. Rank-Based Selection}
	
	\begin{ltrcode}
		\begin{lstlisting}[language=Python, caption=Rank-Based Selection]
			def rank_selection(self, ranked_population):
			n = len(ranked_population)
			# اختصاص rank به جای استفاده از مقدار دقیق fitness
			ranks = list(range(n, 0, -1))  # بهترین = n، بدترین = 1
			total_rank = sum(ranks)
			pick = random.uniform(0, total_rank)
			current = 0
			for i, (chromosome, _) in enumerate(ranked_population):
			current += ranks[i]
			if current > pick:
			return chromosome
		\end{lstlisting}
	\end{ltrcode}
	
	\textbf{مزایا:}
	\begin{itemize}
		\item مستقل از مقیاس شایستگی
		\item کنترل بهتر فشار انتخاب
		\item جلوگیری از dominance زودرس
	\end{itemize}
	
	\textbf{معایب:}
	\begin{itemize}
		\item از دست دادن اطلاعات دقیق شایستگی
		\item محاسبات بیشتر برای رتبه‌بندی
		\item ممکن است همگرایی کندتر باشد
	\end{itemize}
	
	\section{مثال‌های کاربردی}
	
	\subsection{حل سیستم معادلات خطی}
	
	\begin{ltrcode}
		\begin{lstlisting}[language=Python, caption=مثال سیستم خطی دو متغیر]
			# سیستم معادلات:
			# 2x + 3y = 7
			# x - y = 1
			
			equations_text = [
			"2*x + 3*y - 7",
			"x - y - 1"
			]
			
			\begin{ltrcode}
				\begin{lstlisting}[language=Python, caption=مثال سیستم خطی دو متغیر]
					# سیستم معادلات:
					# 2x + 3y = 7
					# x - y = 1
					
					equations_text = [
					"2*x + 3*y - 7",
					"x - y - 1"
					]
					
					solver = EnhancedGeneticEquationSolver(
					pop_size=300,
					generations=1000,
					mutation_rate=0.15
					)
					
					result = solve_equations(equations_text, solver)
					print(f"جواب: x = {result[0]}, y = {result[1]}")
				\end{lstlisting}
			\end{ltrcode}
			
			\subsection{حل سیستم معادلات غیرخطی}
			
			\begin{ltrcode}
				\begin{lstlisting}[language=Python, caption=مثال سیستم غیرخطی]
					# سیستم معادلات:
					# x² + y² = 25
					# x + y = 7
					
					equations_text = [
					"x**2 + y**2 - 25",
					"x + y - 7"
					]
					
					solver = EnhancedGeneticEquationSolver(
					pop_size=500,
					generations=3000,
					mutation_rate=0.2,
					adaptive_mutation=True
					)
					
					result = solve_equations(equations_text, solver)
					print(f"جواب: x = {result[0]}, y = {result[1]}")
				\end{lstlisting}
			\end{ltrcode}
			
			\section{بهینه‌سازی پارامترها}
			
			\subsection{راهنمای تنظیم پارامترها}
			
			\subsubsection{اندازه جمعیت (\lr{Population Size})}
			
			\begin{itemize}
				\item \textbf{مسائل ساده}: 100-300
				\item \textbf{مسائل متوسط}: 300-500
				\item \textbf{مسائل پیچیده}: 500-1000
			\end{itemize}
			
			\subsubsection{نرخ جهش (\lr{Mutation Rate})}
			
			\begin{itemize}
				\item \textbf{شروع}: 0.2-0.3 (اکتشاف بالا)
				\item \textbf{میانه}: 0.1-0.2 (تعادل)
				\item \textbf{پایان}: 0.05-0.1 (بهره‌برداری)
			\end{itemize}
			
			\subsubsection{اندازه تورنمنت (\lr{Tournament Size})}
			
			\begin{itemize}
				\item \textbf{کوچک} (2-3): تنوع بیشتر، همگرایی کندتر
				\item \textbf{متوسط} (4-6): تعادل مناسب
				\item \textbf{بزرگ} (7-10): همگرایی سریع، خطر optimum محلی
			\end{itemize}
			
			\section{خطاهای رایج و راه‌حل‌ها}
			
			\subsection{همگرایی زودرس}
			
			\textbf{علائم:}
			\begin{itemize}
				\item توقف بهبود در نسل‌های اولیه
				\item کاهش شدید تنوع جمعیت
			\end{itemize}
			
			\textbf{راه‌حل‌ها:}
			\begin{itemize}
				\item افزایش اندازه جمعیت
				\item کاهش اندازه تورنمنت
				\item افزایش نرخ جهش
				\item استفاده از restart mechanism
			\end{itemize}
			
			\subsection{همگرایی کند}
			
			\textbf{علائم:}
			\begin{itemize}
				\item بهبود بسیار آهسته fitness
				\item عدم رسیدن به جواب در تعداد نسل‌های تعیین شده
			\end{itemize}
			
			\textbf{راه‌حل‌ها:}
			\begin{itemize}
				\item افزایش اندازه تورنمنت
				\item کاهش نرخ جهش
				\item بهبود تابع شایستگی
				\item استفاده از elitism بیشتر
			\end{itemize}
			
			\section{مطالعات موردی}
			
			\subsection{سیستم معادلات درجه دوم}
			
			\textbf{مسئله:}
			\begin{align}
				x^2 + y^2 &= 13 \\
				x + y &= 5
			\end{align}
			
			\textbf{تحلیل:}
			\begin{itemize}
				\item دو جواب دارد: (2,3) و (3,2)
				\item نیاز به تنوع بالا برای یافتن هر دو جواب
				\item تابع شایستگی مربعی مناسب است
			\end{itemize}
			
			\textbf{پارامترهای بهینه:}
			\begin{ltrcode}
				\begin{lstlisting}[language=Python]
					solver = EnhancedGeneticEquationSolver(
					pop_size=400,
					generations=2000,
					mutation_rate=0.25,
					tournament_size=4,
					elite_size=40
					)
				\end{lstlisting}
			\end{ltrcode}
			
			\subsection{سیستم معادلات با پارامترهای کسری}
			
			\textbf{مسئله:}
			\begin{align}
				\frac{1}{2}x + \frac{1}{3}y &= 1 \\
				\frac{1}{4}x - \frac{1}{6}y &= 0
			\end{align}
			
			\textbf{تحلیل:}
			\begin{itemize}
				\item جواب کسری دارد
				\item نیاز به precision بالا
				\item استفاده از fraction\_precision=True
			\end{itemize}
			
			\textbf{پارامترهای بهینه:}
			\begin{ltrcode}
				\begin{lstlisting}[language=Python]
					solver = EnhancedGeneticEquationSolver(
					pop_size=300,
					generations=1500,
					mutation_rate=0.15,
					fraction_precision=True,
					adaptive_mutation=True
					)
				\end{lstlisting}
			\end{ltrcode}
			
			\section{نتیجه‌گیری}
			
			\subsection{نقاط قوت سیستم}
			
			\begin{enumerate}
				\item \textbf{انعطاف‌پذیری}: قابلیت حل انواع مختلف معادلات
				\item \textbf{مقاومت}: عدم گیر کردن در optimum محلی
				\item \textbf{دقت}: یافتن جواب‌های بسیار دقیق
				\item \textbf{سازگاری}: تطبیق خودکار با انواع مسائل
				\item \textbf{کارایی}: استفاده از تکنیک‌های بهینه‌سازی
			\end{enumerate}
			
			\subsection{محدودیت‌ها}
			
			\begin{enumerate}
				\item \textbf{زمان محاسبه}: نیاز به زمان بیشتر نسبت به روش‌های مستقیم
				\item \textbf{تنظیم پارامتر}: نیاز به دانش برای تنظیم بهینه
				\item \textbf{تضمین همگرایی}: عدم تضمین یافتن جواب در زمان محدود
				\item \textbf{پیچیدگی}: درک و debug کردن سخت‌تر از روش‌های کلاسیک
			\end{enumerate}
			
			\subsection{پیشنهادات برای توسعه آینده}
			
			\begin{enumerate}
				\item \textbf{Parallel Processing}: موازی‌سازی محاسبات
				\item \textbf{Hybrid Methods}: ترکیب با روش‌های عددی کلاسیک
				\item \textbf{Machine Learning}: یادگیری خودکار پارامترها
				\item \textbf{Multi-Objective}: حل همزمان چندین هدف
				\item \textbf{Constraint Handling}: مدیریت بهتر محدودیت‌ها
			\end{enumerate}
			
			\section{منابع و مراجع}
			
			\begin{enumerate}
				\item Goldberg, D. E. (1989). Genetic Algorithms in Search, Optimization, and Machine Learning
				\item Holland, J. H. (1992). Adaptation in Natural and Artificial Systems
				\item Michalewicz, Z. (1996). Genetic Algorithms + Data Structures = Evolution Programs
				\item Coello, C. A. C. (2002). Theoretical and numerical constraint-handling techniques
				\item Deb, K. (2001). Multi-Objective Optimization using Evolutionary Algorithms
			\end{enumerate}
			
			\section{ضمائم}
			
			\subsection{ضمیمه الف: کد کامل کلاس اصلی}
			
			\begin{ltrcode}
				\begin{lstlisting}[language=Python, caption=کلاس کامل EnhancedGeneticEquationSolver]
					import random
					import math
					from fractions import Fraction
					
					class EnhancedGeneticEquationSolver:
					def __init__(self, pop_size=500, generations=5000, mutation_rate=0.2, 
					elite_size=50, var_range=(-100, 100), tournament_size=5,
					stagnation_limit=200, adaptive_mutation=True,
					fraction_precision=True):
					
					self.pop_size = pop_size
					self.generations = generations
					self.base_mutation_rate = mutation_rate
					self.current_mutation_rate = mutation_rate
					self.elite_size = elite_size
					self.var_range = var_range
					self.tournament_size = tournament_size
					self.stagnation_limit = stagnation_limit
					self.adaptive_mutation = adaptive_mutation
					self.fraction_precision = fraction_precision
					
					# آمارها
					self.generation_stats = []
					self.best_fitness_history = []
					self.restart_count = 0
					
					def fitness_function(self, chromosome, equations):
					total_error = 0
					for eq in equations:
					try:
					result = eq(*chromosome)
					total_error += result**2
					except:
					total_error += 1e6
					
					if total_error < 1e-15:
					return float('inf')
					
					return 1.0 / total_error
					
					def initialize_population(self, var_count):
					population = []
					for _ in range(self.pop_size):
					chromosome = []
					for _ in range(var_count):
					strategy = random.choice(['integer', 'small', 'fraction', 'decimal'])
					
					if strategy == 'integer':
					value = random.randint(-10, 10)
					elif strategy == 'small':
					value = random.uniform(-2, 2)
					elif strategy == 'fraction':
					value = random.choice([0.5, 0.25, 0.75, 1.5, 2.5, -0.5, -0.25])
					else:  # decimal
					value = random.uniform(self.var_range[0], self.var_range[1])
					
					chromosome.append(value)
					population.append(chromosome)
					return population
					
					def evaluate_population(self, population, equations):
					evaluated = []
					for chromosome in population:
					fitness = self.fitness_function(chromosome, equations)
					evaluated.append((chromosome, fitness))
					return sorted(evaluated, key=lambda x: x[1], reverse=True)
					
					def selection(self, ranked_population):
					tournament = random.sample(ranked_population, self.tournament_size)
					if random.random() < 0.9:
					return max(tournament, key=lambda x: x[1])[0]
					else:
					sorted_tournament = sorted(tournament, key=lambda x: x[1], reverse=True)
					return sorted_tournament[1][0] if len(sorted_tournament) > 1 else sorted_tournament[0][0]
					
					def crossover(self, parent1, parent2):
					strategy = random.choice(['uniform', 'arithmetic', 'single_point'])
					
					if strategy == 'uniform':
					child = [p1 if random.random() < 0.5 else p2 for p1, p2 in zip(parent1, parent2)]
					elif strategy == 'arithmetic':
					alpha = random.random()
					child = [alpha * p1 + (1 - alpha) * p2 for p1, p2 in zip(parent1, parent2)]
					else:  # single_point
					point = random.randint(1, len(parent1) - 1)
					child = parent1[:point] + parent2[point:]
					
					return child
					
					def mutation(self, chromosome, generation, max_generations):
					if random.random() > self.current_mutation_rate:
					return chromosome
					
					mutated = chromosome.copy()
					gene_index = random.randint(0, len(mutated) - 1)
					
					strategy = random.choices(
					['random', 'small_change', 'fraction', 'zero', 'sign_flip'],
					weights=[0.3, 0.4, 0.2, 0.05, 0.05],
					k=1
					)[0]
					
					if strategy == 'random':
					mutated[gene_index] = random.uniform(self.var_range[0], self.var_range[1])
					elif strategy == 'small_change':
					change = random.gauss(0, 0.1)
					mutated[gene_index] += change
					elif strategy == 'fraction':
					mutated[gene_index] = random.choice([0.5, 0.25, 0.75, 1.0, 2.0, -0.5, -1.0])
					elif strategy == 'zero':
					mutated[gene_index] = random.uniform(-0.1, 0.1)
					elif strategy == 'sign_flip':
					mutated[gene_index] = -mutated[gene_index]
					
					# Update mutation rate
					if self.adaptive_mutation:
					progress = min(1.0, generation / (max_generations * 0.7))
					self.current_mutation_rate = self.base_mutation_rate * (1.0 - 0.6 * progress)
					
					return mutated
					
					def solve(self, equations, max_time=300):
					var_count = len(equations)
					population = self.initialize_population(var_count)
					
					best_solution = None
					best_fitness = 0
					stagnation_counter = 0
					
					for generation in range(self.generations):
					# Evaluate population
					ranked_population = self.evaluate_population(population, equations)
					current_best_fitness = ranked_population[0][1]
					
					# Check for improvement
					if current_best_fitness > best_fitness:
					best_fitness = current_best_fitness
					best_solution = ranked_population[0][0].copy()
					stagnation_counter = 0
					else:
					stagnation_counter += 1
					
					# Check termination conditions
					if best_fitness > 1e10:  # Found exact solution
					break
					
					if stagnation_counter >= self.stagnation_limit:
					# Restart with new population
					population = self.initialize_population(var_count)
					self.restart_count += 1
					stagnation_counter = 0
					continue
					
					# Create next generation
					new_population = []
					
					# Elitism
					for i in range(self.elite_size):
					new_population.append(ranked_population[i][0])
					
					# Generate offspring
					while len(new_population) < self.pop_size:
					parent1 = self.selection(ranked_population)
					parent2 = self.selection(ranked_population)
					child = self.crossover(parent1, parent2)
					child = self.mutation(child, generation, self.generations)
					new_population.append(child)
					
					population = new_population
					
					return best_solution, best_fitness
				\end{lstlisting}
			\end{ltrcode}
			
			\subsection{ضمیمه ب: مثال‌های تست}
			
			\begin{ltrcode}
				\begin{lstlisting}[language=Python, caption=توابع تست مختلف]
					def test_linear_2var():
					equations = [
					"2*x + 3*y - 7",
					"x - y - 1"
					]
					return equations
					
					def test_nonlinear_2var():
					equations = [
					"x**2 + y**2 - 25",
					"x + y - 7"
					]
					return equations
					
					def test_complex_3var():
					equations = [
					"x + y + z - 6",
					"2*x - y + z - 1", 
					"x + 2*y - z - 2"
					]
					return equations
					
					def solve_equations(equation_strings, solver):
					# Parse equations
					equations = []
					for eq_str in equation_strings:
					# Convert to lambda function
					variables = ['x', 'y', 'z', 'w'][:len(equation_strings)]
					eq_lambda = eval(f"lambda {', '.join(variables)}: {eq_str}")
					equations.append(eq_lambda)
					
					# Solve
					solution, fitness = solver.solve(equations)
					
					return solution
				\end{lstlisting}
			\end{ltrcode}
			
			\subsection{ضمیمه ج: جدول پارامترهای بهینه}
			
			\begin{center}
				\begin{longtable}{|p{3cm}|p{2cm}|p{2cm}|p{2cm}|p{3cm}|}
					\hline
					\textbf{نوع مسئله} & \textbf{اندازه جمعیت} & \textbf{نرخ جهش} & \textbf{تعداد نسل} & \textbf{اندازه تورنمنت} \\
					\hline
					\endfirsthead
					
					\hline
					\textbf{نوع مسئله} & \textbf{اندازه جمعیت} & \textbf{نرخ جهش} & \textbf{تعداد نسل} & \textbf{اندازه تورنمنت} \\
					\hline
					\endhead
					
					\hline
					\multicolumn{5}{r}{\textit{ادامه در صفحه بعد...}} \\
					\endfoot
					
					\hline
					\endlastfoot
					
					خطی ساده & 200-300 & 0.15-0.2 & 500-1000 & 3-4 \\
					\hline
					خطی پیچیده & 300-500 & 0.2-0.25 & 1000-2000 & 4-5 \\
					\hline
					غیرخطی ساده & 400-600 & 0.2-0.3 & 1500-3000 & 4-6 \\
					\hline
					غیرخطی پیچیده & 500-800 & 0.25-0.35 & 3000-5000 & 5-7 \\
					\hline
					کسری & 300-500 & 0.15-0.2 & 1000-2000 & 3-5 \\
					\hline
				\end{longtable}
			\end{center}
			
		\end{document}